\documentclass[12pt]{article}

\usepackage{graphicx}
\usepackage{latexsym,algorithm,algorithmic}
\usepackage{times}
\usepackage{amsmath,amssymb,mathtools}
\usepackage{mathptmx}      % use Times fonts if available on your TeX system

\setlength{\textwidth}{7in}
\setlength{\textheight}{9in}
\setlength{\oddsidemargin}{-0.5in}
\setlength{\evensidemargin}{-0.5in}
\setlength{\topmargin}{-0.5in}


\newcommand{\weightvec}{\textbf{w}}
\newcommand{\edge}[2]{{u\rightarrow{}v}}
\newcommand{\edgeuv}{{\edge{u}{v}}}

\newcommand{\vek}[1]{\textbf{#1}}
\newcommand{\ddw}{\frac{\partial}{\partial\vek{w}}}
\newcommand{\M}{\textrm{M}}
\newcommand{\dM}{\textrm{dM}}
\newcommand{\df}{\textbf{df}}
\newcommand{\dt}{\textbf{dt}}
\newcommand{\vphi}{\vec{\phi}}

\begin{document}

\section{Derivation: PPR and its derivative}

Notation: $\vek{s}$ is seeds, $\M$ is transition matrix, $\vek{w}$ are
parameters, and $\vek{p}^\infty$ is the PPR stationary distribution,
$\vek{p}^t$ is a point in the power-rule iteration for computating
PPR, and $\vek{d}^t$ is the partial derivative of $\vek{p}^t$ wrt the
parameters $\vek{w}$.

\begin{eqnarray}
\vek{p}^{t+1} & \equiv & \alpha \vek{s} + (1-\alpha) \M^\top \vek{p}^t \\
\vek{d}^t & \equiv &  \ddw \vek{p}^t
\end{eqnarray}

Note $\vek{p}^t$ and $\vek{d}^t$ have different dimensions: while
$\vek{p}^t_u$ is a scalar score for $u$ under PPR, $\vek{d}^t_u$ is a
vector, giving the sensitivity of that score to each parameter in
$\vek{w}$.

$\M$ is defined as follows.  There is a weight vector $\vek{w}$, and
for an edge $\edgeuv$, there is a feature vector $\vphi_{uv}$, which is
used to define a basic score $s_{uv}$ for the edge, which is passed
thru a squashing function $f$, e.g., $f(x)\equiv e^x$, and then
normalized to form $\M$.

\begin{eqnarray}
 s_{uv}   & \equiv & \vphi_{uv} \cdot \vek{w}  \\
 t_u      & \equiv & \sum_{v'} f(s_{uv'}) \\
 \M_{u,v} & \equiv & \frac{ f(s_{uv}) }{ t_u }
\end{eqnarray}

We can define $\vek{d}^t$ recursively:


\begin{eqnarray}
\vek{d}^{t+1}  & =  &  \ddw \vek{p}^{t+1} \\
               & =  &  \ddw \left( \alpha \vek{s} + (1-\alpha) \M^\top \vek{p}^t \right) \\
               & =  &  (1-\alpha) \ddw \M^\top \vek{p}^t \\
               & =  &  (1-\alpha) \left( (\ddw \M^\top) \vek{p}^t  + \M^\top \ddw\vek{p}^t \right) \\
               & =  &  (1-\alpha) \left( (\ddw \M^\top) \vek{p}^t  + \M^\top \vek{d}^t \right)
\end{eqnarray}

Now let's look at $\ddw\M$, which I'll denote $\dM$ below.  Note that
each $\dM_{uv}$ is a vector, again giving the sensitivity of the
weight $\M_{uv}$ to each parameter in $\vek{w}$.

\begin{eqnarray}
\dM_{uv}   & = & \ddw \frac{ f(s_{uv}) }{ t_u } \\
           & = & \frac{1}{t_u ^2} \left( t_u \ddw f(s_{uv}) - f(s_{uv}) \ddw t_u \right)
\end{eqnarray}

To continue this, define $\df$ and $\dt$ as the vectors

\begin{eqnarray}
 \df_{uv} & \equiv & \ddw f(s_{uv}) = f'(s_{uv}) \vphi_{uv} \label{eqn:fprime} \\
 \dt_{u}  & \equiv & \ddw t_u = \sum_{v'} \df_{uv'}
\end{eqnarray}

Note that $\df_{uv}$ has no more non-zero components than
$\vphi_{uv}$, so it is sparse, and $\dt_{u}$ is also sparse, but
somewhat less so.

We can continue the derivation as

\begin{eqnarray}
\dM_{uv}   & = & \frac{1}{t_u ^2} \left( t_u \ddw f(s_{uv}) - f(s_{uv}) \ddw t_u \right) \\
           & = & \frac{1}{t_u ^2} \left( t_u \df_{uv} - f(s_{uv}) \dt_u \right)
\end{eqnarray}

This is a little different from the old algorithm: there is always an
(implicit) reset with probability $\alpha$, and that reset probability
can be increased by learning by weighting the features for the
(explicit) reset links that are present already in the graph.  No min
$\alpha$!

\section{Computation}

\begin{table} % Computing M and dM
\hrule
\begin{enumerate}
\item For each node/row $u$
  \begin{enumerate}
  \item  $t_u = 0$
  \item  $\dt_u = \vek{0}$, an all-zeros vector
  \item For each neighbor $v$ of $u$
    \begin{enumerate}    
    \item $s_{uv} = \vek{w} \cdot \vphi_{uv}$,  a scalar.

      In detail: For $i\in \vphi_{uv}$: increment $s_{uv}$ by $\vek{w}_i\vphi_i$

    \item $t_u  +\!\!= f(s_{uv})$, a scalar
    \item $\df_{uv} = f'(s_{uv}) \vphi_{uv}$, a vector, as sparse as $\vphi_{uv}$

      In detail: For $i\in \vphi_{uv}$: set $\df_{uv,i} = \vphi_f * c$, where $c=f'(s_{uv})$

    \item  $\dt_u +\!\!= \df_{uv}$, a vector, as sparse as $\sum_{v'} \vphi_{uv'}$

      In detail: For $i\in \df_{uv}$: increment $\dt_{u,i}$ by $\df_{uv,i}$

    \end{enumerate}
  Now $t_u = \sum_{v'} f(s_{uv'})$ and $\dt_{u} = \sum_{v'} \df_{uv'}$
  \item For each neighbor $v$ of $u$ create the vector

\[ \dM_{uv}  = \frac{1}{t_u ^2} \left( t_u \df_{uv} - f(s_{uv}) \dt_u \right)  
\]

Or in detail: For $i\in\dt_u$: $\dM_{uv,i} = \frac{1}{t_u ^2} \left( t_u \df_{uv,i} - f(s_{uv}) \dt_u,i \right)$.  

There aren't any dimensions $i$ that are present in $\df_{uv}$ but not
$\dt_u$, since $\dt_u$ is a summation.  Also create the scalar

\[ \M_{uv}  = \frac{f(s_{uv})}{t_u} 
\]

  You can now discard the intermediate values like $t_u$, $\dt_u$, $\df_{uv}$.

  \end{enumerate}
\end{enumerate}
\caption{Computing $\M$ and $\dM$} \label{alg:mat}
\hrule
\end{table}

Since $\dM$ and $M$ are reused many times, we should compute and store
$M$ and $\dM$ first.  This will expand the size of the graph somewhat:
in particular, we will need to store, not only the active edges $u,v$
and their features $\vphi_{uv}$, but also $\dM_{uv}$, which includes
weights for all features of vertexes $v'$ that are siblings of $v$
(i.e., there is an edge $u,v'$).  I'm not sure how bad this will be in
practice: perhaps we should estimate it for some of our test cases.
After this, operations should run in time linear in the size of the
new, less sparse graph encoded in $\dM$.  I believe that this scheme
also makes operations like the APR learning more modular.

Computing $M$ and $\dM$ is one pass over the graph, shown in Table
\ref{alg:mat}.


Then you can start with $\vek{p}^0 = \vek{d}^0 = \vek{0}$ and iterate

\begin{eqnarray}
\vek{p}^{t+1}  & =  & \alpha \vek{s} + (1-\alpha) \M^\top \vek{p}^t \\
\vek{d}^{t+1}  & =  & (1-\alpha) \left( \dM^\top \vek{p}^t  + \M^\top \vek{d}^t \right)
\end{eqnarray}

In more detail, the iteration for the updates on $\vek{p}$ are shown
in Table~\ref{alg:updates}.

\begin{table} % updating p and d
\hrule

Updating $\vek{p}$:
\begin{enumerate}
\item $\vek{p}^{t+1} = \vek{0}$
\item For each node $u$ 
  \begin{enumerate}
  \item $\vek{p}^{t+1}_u +\!\!= \alpha \vek{s}_u$
  \item For each neighbor $v$ of $u$
    \begin{enumerate}
    \item $\vek{p}^{t+1}_v +\!\!= (1-\alpha) \vek{p}^t_u \M_{uv} $
    \end{enumerate}
  \end{enumerate}
\end{enumerate}

Updating $\vek{d}$:
\begin{enumerate}
\item $\vek{d}^{t+1} = \langle \vek{0}, \ldots, \vek{0} \rangle$ --- i.e., for each node $u$ there is an all-zeros
  vector of weights.
\item For each node $u$ 
  \begin{enumerate}
  \item For each neighbor $v$ of $u$
    \begin{enumerate}
      \item For each $i$ in $\dM_{vu}$

      \[ \vek{d}^{t+1}_{v,i} +\!\!= (1-\alpha) \vek{p}^t_u \dM_{uv,i} \]
      \item For each $i$ in $\vek{d}^t_{v}$

      \[ \vek{d}^{t+1}_{v,i} +\!\!= (1-\alpha) \vek{d}^t_{u,i} \M_{uv} \]
    \end{enumerate}
  \end{enumerate}
\end{enumerate}
\caption{Updates for $\vek{d}$ and $\vek{p}$} \label{alg:updates}
\hrule
\end{table}

\section{Loss functions and lazy regularization}

For SRW each example is a triple $(\vek{s},P,N)$ where $\vek{s}$ is
the seed distribution, $P=\{a^1,\ldots,a^I\}$ are the positive (a-ok?)
examples, and $N=\{b^1,\ldots,b^J\}$ are the negative (bad?) examples.
I use $\vek{p}$ for the PPR distribution starting at $\vek{s}$, and
write $\vek{p}[u]$ for $\vek{p}_u$ if I run out of space for
subscripts.

The loss function is
\begin{equation}
L(\vek{w}) \equiv - \Bigg (\sum_{k=1}^I \log \vek{p}[a^k] + \sum_{k=1}^J \log (1 - \vek{p}[b^k]) \Bigg) + \mu R(\vek{w})
\end{equation}
where $R(\vek{w})$ is the regularization, eg
$R(\vek{w})\equiv||\vek{w}||^2_2$. This means loss increases as the objective function decreases, where the objective function is proportional to the probability of either hitting a positive solution or not-hitting a negative solution.  Once we have $\vek{d}$ it's easy
to compute the gradient of this as

\begin{eqnarray}
\ddw L(\vek{w}) & \equiv & - \Bigg (\sum_{k=1}^I \frac{1}{\vek{p}[a^k]}\vek{d}[a^k] 
                                  - \sum_{k=1}^J \frac{1}{1 - \vek{p}[b^k]}\vek{d}[b^k] \Bigg) + \mu \ddw R(\vek{w})
\end{eqnarray}

We can split this into two parts: the empirical loss gradient, which is

\[
-\Bigg (\sum_{k=1}^I \frac{1}{\vek{p}[a^k]}\vek{d}[a^k] 
     - \sum_{k=1}^J \frac{1}{1 - \vek{p}[b^k]}\vek{d}[b^k] \Bigg)
\]
and the regularization gradient, 
\[
 \mu \ddw R(\vek{w})
\]
If an example doesn't contain all features then the empirical gradient
will be a sparse vector, but the regularization gradient will be
dense.  So the following code might be more efficient for SGD than
just computation of the full gradient and taking a step in that
direction.

\begin{enumerate}
\item Maintain a ``clock'' counter $m$ which is incremented when each
  example is processed. Also maintain a history $\vek{h}_i$ which
  says, for each feature $i$, the last time $t$ an example containing
  $i$ was processed.
\item When a new example $(\vek{s},P,N)$ arrives at time $t$
  \begin{enumerate}
    \item For each feature active in the example, initialize it, if
      necessary, and them perform the regularization-loss gradient
      update $t-\vek{h}_i$ times.
    \item Peform the empirical-loss update, using 
      the new weights.
  \end{enumerate}
\item When you finish learning at final time $T$, consider every
  feature $i$, and perform the regularization-loss gradient update
  $T-\vek{h}_i$ times.  Then write out the final parameters.
\end{enumerate}

\section{Inference: PPR and APR}

Inference using requires only $\M$, which at theorem-proving time is
computed on-the-fly.  That is, whenever you need a row of $\M$ (the
set of values $\M[u,v]$ for all $v$ near $u$) you simply compute the
normalized outlinks of $u$.

The power-iteration version of PPR, which in the codebase is called
the \texttt{PPRProver}, simply iterates this step until convergence
(or for a fixed number of iterations), starting with
$\vek{p}^0=\vek{0}$.

\begin{eqnarray}
\vek{p}^{t+1} & \equiv & \alpha \vek{s} + (1-\alpha) \M \vek{p}^t \\
\end{eqnarray}

Breaking this down, the one-step update is the following.

\begin{enumerate}
\item $\vek{p}^{t+1} = \vek{0}$
\item For each key $u$ with non-zero weight in $\vek{s}$:
  \begin{enumerate}
  \item $\vek{p}^{t+1}[u]$ += $\alpha \vek{s}[u]$
  \end{enumerate}
\item For each key $u$ with non-zero weight in $\vek{p}^t$:
  \begin{enumerate}
  \item For each node $v$ near $u$ in $\M$:
\(
       \vek{p}^{t+1}[v] \mbox{~+=~} (1-\alpha) \M[u,v] \vek{p}^t[u]
\)
  \end{enumerate}
\end{enumerate}

The approximate PageRank is based on a more primitive \texttt{push}
operation.  The full approximate PageRank algorithm also uses
$\vek{d}$, a vector of node degrees.

\begin{enumerate}
\item Let $\vek{p}= \vek{0}$ and $\vek{r}= \vek{s}$
\item While there is some vertex $u$ such that $r(u)\geq{}\epsilon
  \vek{d}(u)$:
  \begin{enumerate}
  \item Perform the \texttt{push}($u$) operation:
    \begin{enumerate}
      \item Save the current value of $u$ in $\vek{r}$: $ru = \vek{r}[u]$
      \item $\vek{p}[u]$ += $\alpha \vek{r}[u]$
      \item $\vek{r}[u]$ *= $(1-\alpha)$
      \item For each node $v$ near $u$ in $\M$:
\(
       \vek{r}[v] \mbox{~+=~} (1-\alpha) \M[u,v] ru
\)
    \end{enumerate}
  \end{enumerate}
\end{enumerate}

In the old implementation, $\vek{d}[u]$ was computed and cached each
time a new key $u$ was added to $\vek{r}$.  In the new implementation
of the prover, this caching is done by the proof graph, so there's no
need to do it in the prover.  

The old DPR prover implementation computed this approximation by
finding $u$'s with large $\vek{r}[u]$ using a variation of the
standard depth-first prover: essentially, you traversed the tree
depth-first, and whenever you hit a node with $\vek{r}[u]$ below
threshold, you stopped.  Factoring this in the algorithm used the same
initial values of $\vek{p}$ and $\vek{r}$ but then called a recursive
routine \texttt{proveState}$(u_0)$, where $u_0$ is the initial query
node (aka the initial state of the proof graph).

William Wang's got a new version which calls this iteratively with
smaller and smaller $\epsilon$'s, which seems to work faster.  That
suggests that it's helpful to do pushes on the nodes with larger
$\vek{r}$ values first.  With that in mind, another thing we could
consider would be storing $u$'s in a heap/priority queue, ordered by
$\vek{r}$ values.  I know this idea has been used in the past but I
don't know how much it helps.  Obviously there's an overhead for the
heap but my guess is that's not going to dominate.

\bigskip

\noindent Function \texttt{proveState}$(u)$:
\begin{enumerate}
\item If $r(u)\geq{}\epsilon \vek{d}(u)$:
  \begin{enumerate}
  \item Perform the \texttt{push}$(u)$ operation
  \item For each $v$ near $u$ (i.e., each subgoal of $u$): \texttt{proveState}$(v)$
  \end{enumerate}
\end{enumerate}


\section{Architectural Comments}

\emph{ [Katie's comments inline and signed with -k]}

\noindent
The routines that are suggested are:
\begin{itemize}
\item Loading: Compute $\M$ and $\dM$ from a graph and $\vek{w}$.  It
  seems necessary, altho a little non-modular, to initialize and
  create new features while the graph is read in. \emph{ [Do this here, or in SGD? -k]} We should be able
  to estimate performance pretty well after this step.

Inputs: graph and current values of parameters $\vek{w}$.

Outputs: $\M$ and $\dM$, and a list of all active features $i$.

Parameters: $f$, $f'$; method to init new features $i$ for $f$, clock
time and lazy-regularization function.
 
\item Extended inference: compute $\vek{p}$ and $\vek{d}$.
 
Inputs: $\vek{s}$, $\vek{w}$, $\M$ and $\dM$.

Outputs: $\vek{p}$ and $\vek{d}$.

\item SGD: Initialize new features $i$, use clock time and
  lazy-regularization function to update their weights, and update
  $\vek{w}$ using the empirical (non-gradient) loss.

Inputs: $\vek{p}$, $\vek{d}$, and $P$, $N$.

Outputs: Modified $\vek{w}$.

\end{itemize}

\begin{table}
\hrule
\begin{verbatim}
class Graph {

  // names of features

  String[] featName;

  // space for labels for feature weights on edges, each is a
  // feature index and a weight

  int[] label_featIndex;
  double[] label_featWeight;

  // space for edges, each is a destination node and a
  // list of labels, which point into the label space

  int[] edge_dst;
  int[] edge_labels_lo;
  int[] edge_labels_hi;

  // space for list of neighbors of nodes, each 
  // points into the edge space

  int [] node_near_lo;
  int [] node_near_hi;

  // list of nodes in graph
  int node_lo, node_hi;

  // space for weight derivatives, which are structurally 
  // just like labels.  these maybe should be vectors since 
  // we don't know the size of this space in advance...?

  int[] deriv_featIndex;
  double[] deriv_featDeriv;

  // list of derivative features for M[u][euv]
  int[][] dM_lo;
  int[][] dM_hi;
}
\end{verbatim}
\caption{Proposed data structure for graph} \label{alg:data}
\hrule
\end{table}

\begin{table}
\hrule
\begin{verbatim}
for (int u = g.node_lo; u < g.node_hi; u++) {
  // euv is edge from u to v
  for (euv = g.node_near_lo[u]; euv < g.node_near_hi[u]; euv++) {
    int v = g.edge_dst[euv];
    // luvk is k-th label on edge euv
    for (luvk = g.edge_labels_lo[euv]; luvk = g.edge_labels_hi[euv]; luvk++) {
      int i = g.label_featIndex[luvk];
      double w = g.label_featWeight[luvk];
      ...
    }
  }
}
for (int d = g.dM_lo[u][euv]; d < g.dM_hi[u][euv]; d++) {
  i = g.deriv_featIndex[d];
  double v = g.deriv_featDeriv[d];
  // v is partial/(partial feature i) if M[u][v]
  ...
}
\end{verbatim}
\caption{Accessing the proposed data structure for graph} \label{alg:access}
\hrule
\end{table}

A proposed data structure which would be very efficient is shown in
Table~\ref{alg:data} and \ref{alg:access}.  Some things that aren't
clear now are:
\begin{itemize}
\item Where does the feature-index $\leftrightarrow$ feature-name
  symbol table go?  Probably not in the graph as shown here.
\item Is $\M$ and $\dM$ part of the graph or a different structure?
\end{itemize}
Also a clarification: \texttt{M} and \texttt{dM\_lo,dM\_hi}, are not
dense 2-matrixes, of size quadratic in the number of nodes, $n$.
Instead \texttt{M} is a length-$n$ array of variable-size arrays, and
\texttt{M[u]}, for node index \texttt{u}, is an array of size $m_u$,
where $m_u$ is the number of neighbors (edges away from) \texttt{u}.
And \texttt{dM\_lo,dM\_hi} are parallel structures.

A final note: the squashing function $f$ and its derivative $f'$ are
only used in computing $\df$ in Equation~\ref{eqn:fprime}.  One future
extension to ProPPR might be to have facts with weights defined by
parameters: e.g., facts like \texttt{sim(a,b)} which when used would
lead to an edge with a computed weight $s_{uv}$ based on an internal
set of parameters $\vec{\lambda}$ (think of a learned similarity
subroutine).  A modular way of dealing with this might be worth
thinking through.  Conceptually, $\vec{\lambda}$ is a part of
$\vek{w}$: one could have an interface which given an edge $u,v$ with
weight $s_{uv}$ returns $f'(s_{uv})\ddw s_{uv}$ in terms of components
of $\vec{\lambda}$.  For training you would also need to pass in an
update of the $\vec{\lambda}$ features to the learning subroutine....

\clearpage
\begin{appendix}
\section{Linear Algebra proof of imperative procedure for p and d updates}

\subsection{Computing the p update}
\begin{align*}
\vek{p}^{t+1} =& \alpha \vek{s} + (1-\alpha)\vek{p}^t \M\\
\vek{p}^t \M =& \left[ \begin{array}{ccccc} 
p_0 & p_1 & p_2 & \ldots & p_n 
\end{array} \right] \left[ \begin{array}{cccc}
\\M_{00} & \M_{01} & \M_{02} & \ldots \\
\\M_{10} & \M_{11} & \M_{12} & \ldots \\
\M_{20} & \M_{21} & \M_{22} & \ldots \\
\vdots & \vdots & \vdots
\end{array} \right]\\
=& \left[ \begin{array}{cccc} p^{t+1}_0 & p^{t+1}_1 & p^{t+1}_2 & \ldots \end{array} \right]\\
p^{t+1}_0 =& p_0 \M_{00} + p_1 \M_{10} + p_2 \M_{20} + \ldots\\
p^{t+1}_1 =& p_0 \M_{01} + p_1 \M_{11} + p_2 \M_{21} + \ldots\\
\end{align*}

If we set $p^{t+1}_u$, we must iterate across, over all terms of the first element, then all terms of the second element, etc:

$$p^{t+1}_u += (1-\alpha) p_v \M_{vu}$$

Given our data structure for $\M$ however, constructing $\M_{vu}$ is difficult. To use $\M_{uv}$ instead, we can set $p^{t+1}_v$ and iterate down, computing the first term of all elements, then the second term of all elements, etc:

$$p^{t+1}_{v} += (1-\alpha) p_u \M_{uv}$$

\subsection{Computing the d update}

Computing $d$ proceeds in much the same way:

\begin{align*}
\vek{d}^{t+1}_i =& (1-\alpha)(\vek{p}^t \dM_i + \vek{d}^t_i \M)
\end{align*}

We'll start by deconstructing the first term only:

\begin{align*}
\vek{p}^t \dM_i =& \left[ \begin{array}{cccc} p_0 & p_1 & p_2 & \ldots \end{array} \right] \
\left[ \begin{array}{cccc}
\dM_{00,i} & \dM_{01,i} & \dM_{02,i} & \ldots \\
\dM_{10,i} & \dM_{11,i} & \dM_{12,i} & \ldots \\
\dM_{20,i} & \dM_{21,i} & \dM_{22,i} & \ldots \\
\vdots & \vdots & \vdots
\end{array} \right]\\
=& \left[ \begin{array}{cccc} d^{t+1}_{0,i} & d^{t+1}_{1,i} & d^{t+1}_{2,i} & \ldots \end{array} \right]\\
d^{t+1}_{0,i} =& p_0 \dM_{00,i} + p_1 \dM_{10,i} + p_2 \dM_{20,i} + \ldots\\
p^{t+1}_{1,i} =& p_0 \dM_{01,i} + p_1 \dM_{11,i} + p_2 \dM_{21,i} + \ldots\\
\end{align*}

We have the same choice of setting $d^{t+1}_{u,i}$ or $d^{t+1}_{v,i}$, with the same dependence on using $\dM_{vu,i}$ or $\dM_{uv,i}$. Since we must use $\dM_{uv,i}$ due to the way our data structure is constructed, we must update $d^{t+1}_{v,i}$:

$$ d^{t+1}_{v,i} += (1-\alpha) \vek{p}^t_u \dM_{uv,i} $$

The same principles hold for the second term:

$$ d^{t+1}_{v,i} += (1-\alpha) \vek{d}^t_{u,i} \M_{uv} $$

\end{appendix}

\end{document}

